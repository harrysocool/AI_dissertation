%!TEX root = ../thesis.tex

\chapter{Results}
This chapter will include all the results produced during this project experiment. 
It included the plot of time usage under each method from section 2.
As for the simulate of real world scenario under database image, we obtained two ways which are Gaussian Noises and Partial Occlusions. 
Different inspection methods were applied for the accuracy, robust and reliability comparison between each methods.
%====================================================
%
%====================================================
\section{Object Proposal Method Tuning}
For each object proposal method, there are some parameters needed to be chosen in order to perform better detection accuracy.
In order to evaluate the performance of object proposal method, we use a method called Intersection of Unit (IOU) as shown in figure \ref{IOU_1} below:
\fig{IOU_1}{Illustration of the measurement of Intersection of Unit (IOU)}{0.5} 

Because the performance of object proposal method in highly related to the number of proposal boxes which contains ear (high IOU with ground truth box).
We calculate the IOU for every proposals and count only the one with $IOU >= 0.1$ then divided by total number of proposals to get the Potential Proposals Ratio (PPR).
%====================================================
%
%====================================================
\section{Time Usages}
As for the three Object Proposal methods we introduced, the former two algorithms Selective Search (SS) and Edge Boxes Detector (ED) was released only in MATLAB version, but the Binarized Normed Gradients (BING) was in C++. 
Therefore, BING will be faster with respect to the data exchange between MATLAB and the neural network will be reasonably slower.

By the record of a survey produced by Jan Hosang et al. \cite{Hosang:2014um}, the approximate time of these algorithms are 10, 0.3 and 0.2 seconds for SS, ED and BING. 
However it is slightly different under our hardware specification \improvement{appendix ref} as measured and plotted below in figure \ref{result_time_1.eps}.
We randomly chosen 200 images for the time measurement.
\begin{itemize}
\itemsep=-0.5em
  \item For \textbf{SS}: it takes about 1.3 seconds for average 400 object proposals to be calculated. 
  \item For \textbf{ED}: 0.5 seconds in average when producing the least proposals at 200.
  \item For \textbf{BING}: the number of proposals is fixed to 800 per image and the time is almost the same as ED at 0.5 seconds.
\end{itemize}

\fig{result_time_1.eps}{The time usage and corresponding object proposals}{0.5}

As for the time usage in the CNN part is highly related to the amount of object proposals. 
Hence after several attempts and measurements, the average times for the 600*800 colour image in the SOTON database for those 3 methods SS, ED and BING are respectively $\textbf{1.75, 0.73 and 1.05 seconds}$. 
%====================================================
%
%====================================================
\section{Detection Rate Measurement}
This section shows the results of the accuracy when performing this ear detection algorithm.
Firstly, it described the conditions for true positive detection and present the result when performing it on testing dataset. 
Then the noise and occlusion were applied to test the robustness and reliability of this algorithm.
%============================================================
\subsection{Conditions for Positive Detection}
As it shown in figure \ref{result_1} (a), the fast-RCNN will produces many boxes along with its probability to the ear.
Therefore, we apply a simple threshold to remove the boxes with $P(ear|box)<0.8$. 
Then the remaining boxes was filtered by a Non Maximum Suppression method for only one bounding box left.
However, sometimes it may left more than one boxes due to false positive detection.

\fig{result_1}{(a) All output boxes and probability from fast-RCNN (b) threshold by $P(ear|box)>0.8$ (c) Non Maximum Suppression of left boxes (d) final result with IOU}{0.95}

We use another method to measure the false positive detection, which is the Intersection of unit rate.
It indicate how many area this predicting box overlapped with the ground truth box as illustrated in figure \ref{IOU_1}.
If the IOU rate was less than 0.5, then we determine it as a false positive detection. 
\vspace{-4mm}
\begin{itemize}
\itemsep=-0.5em 
  \item \textbf{Positive Detection}: has more than one box with $P(ear|box)>0.8$.
  \item \textbf{False Positive Detection}: fulfill requirement for positive detection but the biggest IOU is less than 0.5
  \item \textbf{Negative Detection}: no boxes with $P(ear|box)>0.8$.
\end{itemize}


\fig{IOU_1}{Illustration of the measurement of Intersection of Unit (IOU)}{0.5} 
%============================================================
\subsection{Gaussian Noise Distraction}
In this section, we add random Gaussian noises into the raw images with 0 mean and variational $\sigma$ from 5 to 30 to stimulate the noises in real world scenario such as CCTV surveillance images. 
The sample images can be seen from figure \ref{noise_1} below. 
It illustrates how severe the noises are with different scale of $\sigma$.
\fig{noise_1}{Gaussian noise distraction samples by ED method}{0.95}

\figtwo{noise_DR_1}{True positive detection rate under different Gaussian $\sigma$}{0.9}{noise_FPR_1}{False positive detection rate under different Gaussian $\sigma$}{0.9}
\fig{noise_IOU_1}{Average of IOU rate under different Gaussian $\sigma$}{0.5}

Three result parameters were chosen as the performance indicators to help compare these 3 object proposal methods. 
They are well explained in section 3.2.1.

For the detection rate, it is plotted in figure \ref{noise_DR_1} and \ref{noise_FPR_1} showing good accuracy even for $\sigma = 30$ with more than 70\% positive detection rate via ED object proposal method.
The BING method, even it provided the most object proposal boxes up to 800 per images, the detection accuracy decreases almost linearly with the increase of noise ratio.
It only reached beyond 90\% in the raw images and with $\sigma = 5$ Gaussian noise. 

Surprisingly, both BING and SS methods have an increase of detection accuracy along the increase of noise in the beginning.
It might be because of the Gaussian noises change the number and locations of object proposal boxes, and accidentally triggered some positive detections.

On the other hand, all 3 methods produced good reliability along the increase of Gaussian noises. 
All of the False Positive Detection rate remains below 10\% except for BING at $\sigma = 30$.
Another result in figure \ref{noise_IOU_1} also proved good reliability, which states that the average IOU rate of all the positive ear detection by these methods are above 70\%.

%============================================================
\subsection{The Use of Partial Occlusion}
Usually ear will be covered by either hair or some kind of earrings, therefore a robust ear detection algorithm must be able to detect ear under that circumstances.
Therefore we simulate the occlusion of ear by put a black mask onto the ground truth ear box.
The mask size was determined both by the ear box size and the parameter called "occlude percentages".
As it shown below in figure \ref{occlude_1}, we test from 10\% to 50\% occlusion, found that 50\% occlusion was very difficult for our algorithm to perform detection.
\fig{occlude_1}{Partial occlusion samples by ED method}{0.95}

The detection rate drops dramatically when the occlusion percentages hit 50\%, however, the ED method can have more than 90\% accuracy when dealing less than 50\% occlusion with less than 20\% false positive detection.
The BING method only performs well when occlusion percentages less than 30\%, reaches more than 90\% detection accuracy.
However, it decreases exponentially after 30\% with the rise of false positive detection rate.
The SS method performs normally from 60\% in raw images to 10\% in 50\% occlusion images. 

\figtwo{occlude_DR_1}{True positive detection rate under different occlusion percentage}{0.9}{occlude_FPR_1}{False positive detection rate under different occlusion percentage}{0.9}

\fig{occlude_IOU_1}{Average of IOU rate under different occlusion percentage}{0.5}

Despite of the low detection rate with SS method, the average IOU of it reaches the same level as the other two methods. It only drops from 80\% to 50\% alongside the increase of occlusion percentages.