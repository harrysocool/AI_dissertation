%!TEX root = ../thesis.tex
\begin{appendices}

\chapter{Hardware \& Software Specification}
\label{app:appendixA}
The Training and testing of this project was undertaking by a LENOVO Y470 laptop which has specification below. 
The reason for not using high performance computer in the Computer Lab is the graphic driver of the lab computer is incompatible with the installation of fast-RCNN network.

\begin{itemize}
\itemsep=-0.5em
  \item \textbf{Purchase Year}: September 2011
  \item \textbf{Operation System (OS)}: Ubuntu 15.04
  \item \textbf{CPU}: Intel Core i5-2540M @2.60GHz x 4
  \item \textbf{GPU}: Nvidia GeForce GT550M
  \item \textbf{GPU Memory}: 1024MB
  \item \textbf{RAM}: 6GB
\end{itemize}

The main part of the whole project is written in Python and can be easily downloaded from Internet at website \url{https://github.com/harrysocool/ear_recognition.git} (exclude the ear database).
The version of python is 2.7.11, and the three object proposal methods are also modified and kept in the Github repository.
The training and testing script of fast-RCNN are highly modified for compatible with our own database.
\end{appendices}
