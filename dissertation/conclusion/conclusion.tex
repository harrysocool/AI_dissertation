%!TEX root = ../thesis.tex
\chapter{Discussion \& Conclusion}
\label{ch:conclusions}
The objective for this project is to apply deep learning method into ear detection and achieve a good detection rate under different circumstances. 
In the section before, result proves the feasibility of such an algorithm by combining object proposal method with fast Region Convolutional Neural Network.
Although the time for our algorithm is still more than 500ms per image which unable to perform a live detection, the old laptop we used is responsible for that.
It can be easily solved by using a higher performance computer.

\section{Summary of Project Achievements}
In the past decades, the classic hand-craft feature matching method was dominating the ear detection, or even object detection fields. 
It is a competition of finding the most useful features to representing ear and highlight it in an image.
Many researchers have found their way to achieve rank-1 detection rate over 90\% or even 100\% on their chosen database \cite{Pflug:2012cja}.
However, it is the generalization and robustness problem that prevent this technique from widely used in real world applications.

Along with the explosion of computer performance and the successful application of Convolutional Neural Network (CNN) into object detection and classification recently.
Although CNN was invented in 2012 by Hinton et al. \cite{Krizhevsky:2012wl}, it boosted the development in many fields not only for computer vision.
Figure \ref{convnet_1} illustrates that it only took 2 years for CNN to almost doubled the mean Average Precision in PASCAL VOC object detection challenge.
Theoretically, ear detection is an object detection method specifically for ear. 
Therefore it should be easy to combined with the state-of-art CNN method in order to perform such a complexity-reduced problem, yet no record can be found on the Internet that anyone has attempt this before. 

This project successfully bring the state-of-art method fast-RCNN into the first step of ear biometrics, ear detection, and achieved a rather good rank-1 100\% detection accuracy.
We attempt to inspire the following researchers by 


\section{Applications}

Applications.


\section{Future Work}

Future Work.
