%!TEX root = ../thesis.tex

\chapter{Introduction}

\section{Motivation and Objectives}

Motivation and Objectives here.

\section{Literature Reviews}
\subsection{The Early Work towards Ear Forensics by Iannarelli}
In 1949, Alfred Iannarelli was well-known in American as an ear identification export who providing ear evidence as a strong personal identification in the context of forensic science\cite{State:1999}. He also developed a measurement system in order to representing the ear by numbers, which was used by American law enforcement agencies at that time\cite{Arbab-Zavar:2009}. As shown in Figure \ref{Iannarel_ear}, he aligned the ear into 4 reference lines which divided the image into equally 45\degree\ intervals. The intersection of those lines is located on the tragus and the crus of helix (see section 4.1 for a description of the anatomical parts of the ear)\unsure{Need change the references}, therefore the ear image can be represented based on the edge markers of each reference line.

\fig{Iannarel_ear}{Iannarelli’s manual ear measurement system.}{0.3} 

According to Iannarelli's book, through 38 years of research and application in earology on more than 10,000 ear images, no two ears were found to be identical, not even the two ears from the same person\cite{Arbab-Zavar:2009}.
\unsure{Maybe more conclusion}
\par Although this system was based on a man-power visual measurement which may be not very accurate, the usage time of the successful application has proved itself to be very useful. It also inspired a lot of researches towards ear recognition using more advanced technology later.

\subsection{Burge and Burgers' method}
In the early time of the 21st Century, Burge and Burger started to use machine vision for ear recognition system. They were the first who managed to use machine model each individual ear with an adjacency graph.
\par The main step of their method can be shown as figure \ref{burge_ear}. They used the canny edge detection firstly to extract the "ear print", then reconstructed it into a Voronoi diagram which looks like a segmentation of the ear parts. Finally, join the center of each segment to form the "N-graph" which is used to authenticating a person.

\fig{burge_ear}{Burge and Burgers' ear model method.}{0.55}

\subsection{Shaped Wavelets for Ear Detection}
Due to the ear image mainly contains a lot of curvilinear structures, Ibrahim et al.\cite{Ibrahim:2010dc} convolve the image with some curved wavelet filters called "banana wavelet" shown as the figure \ref{banana_1} to perform a generalized template matching to detect the location of ear.

\fig{banana_1}{Banana wavelets used in this method.}{0.55}

\par The Banana wavelets are a generalization of Gabor wavelets and it can be parameterized by four variables: frequency, curvature, orientation and size. They use 8 filters which prove to be sufficient for ear detection as it shown above in figure \ref{banana_1}. Initially, they do the convolution between image and the banana wavelet which resulting the magnitude of the filter response. Then the local maxima of the magnitude should be the position where ear has similar curvature, size and orientation to the specific corresponding banana wavelets, as it shown below in figure \ref{banana_2}. \EE

\fig{banana_2}{(a) Input image, and (b)-(i) after convolution with 8 banana filters}{0.65}

