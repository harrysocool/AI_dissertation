%!TEX root = ../thesis.tex

\chapter{Introduction}
Human ear, as a very promising biometric identifier has drew lots of attentions in biometric communities recently \cite{Ibrahim:2011dq}. It contains sufficient curved structures which will not change radically from age 8-70 years old and are unaffected by cosmetics, unlike face \cite{Burge:2000ek}. Compare to gait, it cannot be easily changed intentionally by the subject when doing a surveillance. Which makes it a perfect and promising biometric for access control, security and video surveillance.

Recently, people trends to focus on using 3D data or combine 2D with 3D data due to it can overcome the illumination and pose variation of ear recognition, although it can get much better result now, special equipment and expensive computation will be needed \cite{Yuan:2012db}. However, this manuscript will concentrate on 2D images, because of the consistency when deploy in surveillance or other planar image scenarios \cite{ArbabZavar:2011cn}.
%====================================================
%
%====================================================
\section{Motivation and Objectives}
The main motivation for ear recognition would be the implementation of an algorithm which can automatically tell the identity of a person via only taking pictures of his ear. Such a system can then be used for security surveillance, collect crime evidences and attendance check. Combined with the state-of-art face recognition system, they can provide more accurate identification rate and be more reliable. 

For an automatic ear recognition system, it contains three main parts: ear detection, feature extraction and identification (classification). Due to the limitation of time in this project, it is better to focus on one part out of all three. Therefore, this project will concentrate on the first part \textbf{ear detection} which has very significant impact to the following procedures. For example, if the ear failed to be registered inside an image, the following steps tend to be meaningless and may cause incorrect detection rate data. 

Therefore, the objective of this project can be listed below:
\vspace{-4mm}
\begin{itemize}
\itemsep=-0.5em
  \item Implement an algorithm which has good detection rate on ear images under some circumstances such as occlusion and noises which often occur in the real world scenario. 
  \item Attempt to use state-of-art deep-learning and convolutional network methods instead of classic hand-craft feature spaces.
  \item Reduce the time for the detection so that it can be used as a live detection.
\end{itemize}
\improvement{objective}
%====================================================
%
%====================================================
\section{Outline}
\unsure{.}
This manuscript is divided into 4 chapters which mainly covered all the informations required. 
Chapter 1 is the introduction of the whole picture followed by the literature review comparing classic method on ear detection applications. 
Chapter 2 introduce the methodology employed in this project including the platform specification and a brief introduction of the environment setup. 
Chapter 3 shows the result of this method under different restrictions as it usually happened in the real world. 
The final chapter will discuss the utility of result in the previous chapter, and perform an analysis under that.
%====================================================
%
%====================================================
\section{Literature Reviews}
In this section, the previous work in ear detection is presented with descriptive words. It was mainly about the classic methods which focus more on how to design the feature space then scan the whole image to match such features. However, we are implementing a novel method for ear detection which no one has done before, no related paper can be discussed in this section about it.

\subsection{Reduced Hough Transform for Ear Detection}
Hough Transform (HT) is a very classic algorithm in image processing widely used for feature extraction and patten recognition. It is useful to find the imperfect instances of objects in certain shapes which is very suitable for ear detection as the ear is just like a ellipse and remains that way. Hence, in 2007 Arbab-Zavar et al.\cite{ArbabZavar:2007hr} used it to design an algorithm for automatic ear detection. Despite the advantages, HT has certain drawbacks such as high computational requirement and memory usages, therefore they used a reduced Hough Transform to overcome those problems as it was specified for only detect ellipse using the known geometrical properties of it to decompose the parameter space from 5D to 2D.

\fig{arbab_1}{The process of Arbab-Zavar's method \cite{ArbabZavar:2007hr}. (b)Canny edge (c)accumulator of ellipse (d)reduce the horizontal vote}{0.9}

Firstly, a Canny operator is applied to get a smoothed edge detected image, then the reduced HT transform reconstruct it into an accumulator space as it shown in figure \ref{arbab_1} (c). The locations of the peaks will provide the coordination of the best matching ellipse. However, there are some mismatches as well, such as the presence of the spectacles. The way to eliminate most of them is to get rid of the horizontal vote by HT, due to the ear shape is mostly a vertical ellipse in the database. 

The result shows that this algorithm achieves error-free on the XM2VTS derived, 252-image database. But not that good in UND database, given that it include more backgrounds informations. When applying occlusions, despite the HT was known as tolerate to noise and imperfect, the detection accuracy still drops below 80\% after 40\% occlusion percentages and reaches 30\% at 70\% partial occlusion \cite{ArbabZavar:2007hr}. This algorithm can be seen as very successful under the condition of head profile only database, without backgrounds and large hair covers. It is a very good example of how to maximize the usage of classic methods, however the speed of detection needs to be evaluated if the algorithm needs to be improved.
%=========================================================

\subsection{Haar-like Features for Ear Detection} 
The Haar-like feature was a very successful algorithm when applied for human face detection by Viola at 2001\cite{Viola:2001un}. Therefore, in 2009 Yuan et al.\cite{Yuan:2009js} designed an algorithm which used Haar-like features for ear detection. There are some extended asymmetric Haar-like features as in figure \ref{Yuan_1} were added as the ear structure is different from face.

\figtwo{Yuan_1}{Haar-like features reflects the local features}{0.5}{Yuan_2}{The procedure of classify ears}{0.8}

They trained several strong classifiers with AdaBoost algorithm and then cascaded them together into a multi-layer classifier \cite{Yuan:2009js}. As all the sub-windows of the image pass through each one of the single classifier, false response will immediately reject the corresponding sub-window. Only the one went through all classifiers will be marked as an ear.

The training sample came from their own database USTB \unsure{database references} including 11,000 images with half left ear and half right. There are also 10,000 negative samples from CAS-PEAL face databases. The result was very impressive with only 0.5\% False Reject Rate and 2.3\% False Acceptance Rate on USTB 220 testing images. However this method will be highly related to the quality of training dataset and require good control of over-fitting.

\subsection{SIFT Point Matching for Ear Detection}
Scale-invariant feature transform also known as SIFT is a famous computer vision algorithm for detecting and describing local features in images. It extract the interest point of the object in an image to provide "feature description" of the object, then a chosen distance can be used to decide if two description are the same between two images. Advantages of the SIFT for ear detection is that it is scale-invariant and not sensitive with noise and illumination.

Based on the method written by Brown et al.\cite{Brown:2002is}, they try to created a homography transform between a probe image and a known gallery object image using SIFT matches. If an homography can be created, means that the probe contains the gallery object. In addition, 4 matched SIFT points were used to align if they lies in one plane , due to the unreliability of more points. Although it can provided an accurate result, a RANSAC algorithm was used to select the best match.

The detection result was under the XM2VTS derived, 252-image database \unsure{database references} shown that it achieved 96\% rank-1 detection accuracy \cite{Bustard:2008ha}. However it requires the predefined galley of object image which needs a manual mask to locate the object.

\subsection{Shaped Wavelets for Ear Detection}
Due to the ear image mainly contains a lot of curvilinear structures, Ibrahim et al.\cite{Ibrahim:2010dc} convolve the image with some curved wavelet filters called "banana wavelet" shown as the figure \ref{banana_1} to perform a generalized template matching to detect the location of ear.

\fig{banana_1}{Banana wavelets used in this method\cite{Ibrahim:2010dc} .}{0.55}

The Banana wavelets are a generalization of Gabor wavelets and it can be parameterized by four variables: frequency, curvature, orientation and size. They use 8 filters which prove to be sufficient for ear detection as it shown above in figure \ref{banana_1}. Initially, they do the convolution between image and the banana wavelet which resulting the magnitude of the filter response. Then the local maxima of the magnitude should be the position where ear has similar curvature, size and orientation to the specific corresponding banana wavelets, as it shown below in figure \ref{banana_2}. Finally, some threshold and anti-overlapping algorithms can be applied to make the decision where is the ear. 

\fig{banana_2}{(a) Input image, and (b)-(i) after convolution with 8 banana filters\cite{Ibrahim:2010dc}}{0.65}

The results demonstrate that this is a very promising methods which achieve 100\% detection rate on the XM2VTS database and above 98\% when the Gaussian noise ($\sigma = 100$) is presented. However, when testing on the SOTON database \unsure{database references} with some occlusion, the detection accuracy drop dramatically to 44.7\% with partial head and small occlusion\cite{Ibrahim:2010dc} . 

Although the algorithm treat the ear as the combination of some curved lines and focus on finding those lines, it cannot achieve better detection rate when occlusion occurs. The amount of calculation must be very big due to the several fully convolution of whole image with 8 wavelet filters, but the author did not mention anything about speed. 

\subsection{Active Contour for Ear Detection}
In 2011, Kumar et al.\cite{Kumar:2011ci} tried to develop an online application which can use ear biometric as an authentication method. They build a wooden stand for acquiring the image as it shown in figure \ref{kumar_1}, so there are no extra illumination management require, the ear was captured inside a isolated box with utilized camera flash light. Which should make the detection more easy, however can not handle the real scenario image.

\fig{kumar_1}{(a)Imaging setup (b)Sample captured image}{0.55}

First of all, they use Gaussian classifier to detect the skin region, then apply LoG(Laplacian of Gaussian) for edge detection and remove the false edge until only ear edge left. By using the top and bottom pixel of the ear edge, they can manage to rotate the ear with reference to the vertical axis. Finally, a localized region based active contour model is applied to extract the ear part only from the ear-ROI(Region of Interest). The whole process is shown in figure \ref{kumar_2}.

\fig{kumar_2}{(a)Ear Edge by LoG (b)False edge removal (c)ear-ROI (d)Ear-contours}{0.55}

The database they created is 100 users with 7 pictures each. However, the detection rate from the ear-ROI is about 94.2\%, which means the correct ear-contours extract is 660 images, the blurry images and the presence of hair are the ones failed \cite{Kumar:2011ci}. As they only uses the specific controllable database, it is no doubt that this method will failed more when applying in real life images.

\subsection{Convolutional Neural Network}
\fig{convnet_1}{Historical PASCAL VOC object detection rate}{0.6}
\fig{convnet_2}{Architecture of the Convolutional Neural Network}{0.8}

\section{Contribution}
Due to the implementation of convolutional neural network on object detection was just few years ago, no one has used it on ear image database, therefore no related work published.

