%!TEX root = ../thesis.tex

\chapter{Introduction}

\section{Motivation and Objectives}

Motivation and Objectives here.

\section{Literature Reviews}
\subsection{The Early Work towards Ear Forensics by Iannarelli}
In 1949, Alfred Iannarelli was well-known in American as an ear identification export who providing ear evidence as a strong personal identification in the context of forensic science\cite{State:1999}. He also developed a measurement system in order to representing the ear by numbers, which was used by American law enforcement agencies at that time\cite{Arbab-Zavar:2009}. As shown in Figure \ref{Iannarel_ear}, he aligned the ear into 4 reference lines which divided the image into equally 45\degree\ intervals. The intersection of those lines is located on the tragus and the crus of helix (see section 4.1 for a description of the anatomical parts of the ear)\unsure{Need change the references}, therefore the ear image can be represented based on the edge markers of each reference line. 
\begin{figure}[!h]
  \centering
  \includegraphics[width=0.3\textwidth]{figures/Iannarel_ear.png}
  \caption{Iannarelli’s manual ear measurement system.}
  \label{Iannarel_ear}
\end{figure}
According to Iannarelli's book, through 38 years of research and application in earology on more than 10,000 ear images, no two ears were found to be identical, not even the two ears from the same person\cite{Arbab-Zavar:2009}.
\unsure{Maybe more conclusion}
\par Although this system was based on a man-power visual measurement which may be not very accurate, the usage time of the successful application has proved itself to be very useful. It also inspired a lot of researches towards ear recognition using more advanced technology later.

\subsection{Burge and Burgers' method}
In the early time of the 21st Century, Burge and Burger started to use machine vision for ear recognition system. They were the first who managed to use machine model each individual ear with an adjacency graph.
\begin{figure}[!h]
  \centering
  \includegraphics[width=0.7\textwidth]{figures/burge_ear.png}
  \caption{Burge and Burgers' ear model method.}
  \label{burge_ear}
\end{figure}
